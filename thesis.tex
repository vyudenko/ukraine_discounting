% Options for packages loaded elsewhere
\PassOptionsToPackage{unicode}{hyperref}
\PassOptionsToPackage{hyphens}{url}
\PassOptionsToPackage{dvipsnames,svgnames,x11names}{xcolor}
%
\documentclass[
  letterpaper,
  DIV=11,
  numbers=noendperiod]{scrartcl}

\usepackage{amsmath,amssymb}
\usepackage{iftex}
\ifPDFTeX
  \usepackage[T1]{fontenc}
  \usepackage[utf8]{inputenc}
  \usepackage{textcomp} % provide euro and other symbols
\else % if luatex or xetex
  \usepackage{unicode-math}
  \defaultfontfeatures{Scale=MatchLowercase}
  \defaultfontfeatures[\rmfamily]{Ligatures=TeX,Scale=1}
\fi
\usepackage{lmodern}
\ifPDFTeX\else  
    % xetex/luatex font selection
\fi
% Use upquote if available, for straight quotes in verbatim environments
\IfFileExists{upquote.sty}{\usepackage{upquote}}{}
\IfFileExists{microtype.sty}{% use microtype if available
  \usepackage[]{microtype}
  \UseMicrotypeSet[protrusion]{basicmath} % disable protrusion for tt fonts
}{}
\makeatletter
\@ifundefined{KOMAClassName}{% if non-KOMA class
  \IfFileExists{parskip.sty}{%
    \usepackage{parskip}
  }{% else
    \setlength{\parindent}{0pt}
    \setlength{\parskip}{6pt plus 2pt minus 1pt}}
}{% if KOMA class
  \KOMAoptions{parskip=half}}
\makeatother
\usepackage{xcolor}
\usepackage{soul}
\setlength{\emergencystretch}{3em} % prevent overfull lines
\setcounter{secnumdepth}{-\maxdimen} % remove section numbering
% Make \paragraph and \subparagraph free-standing
\ifx\paragraph\undefined\else
  \let\oldparagraph\paragraph
  \renewcommand{\paragraph}[1]{\oldparagraph{#1}\mbox{}}
\fi
\ifx\subparagraph\undefined\else
  \let\oldsubparagraph\subparagraph
  \renewcommand{\subparagraph}[1]{\oldsubparagraph{#1}\mbox{}}
\fi


\providecommand{\tightlist}{%
  \setlength{\itemsep}{0pt}\setlength{\parskip}{0pt}}\usepackage{longtable,booktabs,array}
\usepackage{calc} % for calculating minipage widths
% Correct order of tables after \paragraph or \subparagraph
\usepackage{etoolbox}
\makeatletter
\patchcmd\longtable{\par}{\if@noskipsec\mbox{}\fi\par}{}{}
\makeatother
% Allow footnotes in longtable head/foot
\IfFileExists{footnotehyper.sty}{\usepackage{footnotehyper}}{\usepackage{footnote}}
\makesavenoteenv{longtable}
\usepackage{graphicx}
\makeatletter
\def\maxwidth{\ifdim\Gin@nat@width>\linewidth\linewidth\else\Gin@nat@width\fi}
\def\maxheight{\ifdim\Gin@nat@height>\textheight\textheight\else\Gin@nat@height\fi}
\makeatother
% Scale images if necessary, so that they will not overflow the page
% margins by default, and it is still possible to overwrite the defaults
% using explicit options in \includegraphics[width, height, ...]{}
\setkeys{Gin}{width=\maxwidth,height=\maxheight,keepaspectratio}
% Set default figure placement to htbp
\makeatletter
\def\fps@figure{htbp}
\makeatother
\newlength{\cslhangindent}
\setlength{\cslhangindent}{1.5em}
\newlength{\csllabelwidth}
\setlength{\csllabelwidth}{3em}
\newlength{\cslentryspacingunit} % times entry-spacing
\setlength{\cslentryspacingunit}{\parskip}
\newenvironment{CSLReferences}[2] % #1 hanging-ident, #2 entry spacing
 {% don't indent paragraphs
  \setlength{\parindent}{0pt}
  % turn on hanging indent if param 1 is 1
  \ifodd #1
  \let\oldpar\par
  \def\par{\hangindent=\cslhangindent\oldpar}
  \fi
  % set entry spacing
  \setlength{\parskip}{#2\cslentryspacingunit}
 }%
 {}
\usepackage{calc}
\newcommand{\CSLBlock}[1]{#1\hfill\break}
\newcommand{\CSLLeftMargin}[1]{\parbox[t]{\csllabelwidth}{#1}}
\newcommand{\CSLRightInline}[1]{\parbox[t]{\linewidth - \csllabelwidth}{#1}\break}
\newcommand{\CSLIndent}[1]{\hspace{\cslhangindent}#1}

\KOMAoption{captions}{tableheading}
\makeatletter
\makeatother
\makeatletter
\makeatother
\makeatletter
\@ifpackageloaded{caption}{}{\usepackage{caption}}
\AtBeginDocument{%
\ifdefined\contentsname
  \renewcommand*\contentsname{Table of contents}
\else
  \newcommand\contentsname{Table of contents}
\fi
\ifdefined\listfigurename
  \renewcommand*\listfigurename{List of Figures}
\else
  \newcommand\listfigurename{List of Figures}
\fi
\ifdefined\listtablename
  \renewcommand*\listtablename{List of Tables}
\else
  \newcommand\listtablename{List of Tables}
\fi
\ifdefined\figurename
  \renewcommand*\figurename{Figure}
\else
  \newcommand\figurename{Figure}
\fi
\ifdefined\tablename
  \renewcommand*\tablename{Table}
\else
  \newcommand\tablename{Table}
\fi
}
\@ifpackageloaded{float}{}{\usepackage{float}}
\floatstyle{ruled}
\@ifundefined{c@chapter}{\newfloat{codelisting}{h}{lop}}{\newfloat{codelisting}{h}{lop}[chapter]}
\floatname{codelisting}{Listing}
\newcommand*\listoflistings{\listof{codelisting}{List of Listings}}
\makeatother
\makeatletter
\@ifpackageloaded{caption}{}{\usepackage{caption}}
\@ifpackageloaded{subcaption}{}{\usepackage{subcaption}}
\makeatother
\makeatletter
\@ifpackageloaded{tcolorbox}{}{\usepackage[skins,breakable]{tcolorbox}}
\makeatother
\makeatletter
\@ifundefined{shadecolor}{\definecolor{shadecolor}{rgb}{.97, .97, .97}}
\makeatother
\makeatletter
\makeatother
\makeatletter
\makeatother
\ifLuaTeX
  \usepackage{selnolig}  % disable illegal ligatures
\fi
\IfFileExists{bookmark.sty}{\usepackage{bookmark}}{\usepackage{hyperref}}
\IfFileExists{xurl.sty}{\usepackage{xurl}}{} % add URL line breaks if available
\urlstyle{same} % disable monospaced font for URLs
\hypersetup{
  pdftitle={My Manuscript},
  colorlinks=true,
  linkcolor={blue},
  filecolor={Maroon},
  citecolor={Blue},
  urlcolor={Blue},
  pdfcreator={LaTeX via pandoc}}

\title{My Manuscript}
\author{}
\date{}

\begin{document}
\maketitle
\ifdefined\Shaded\renewenvironment{Shaded}{\begin{tcolorbox}[boxrule=0pt, interior hidden, enhanced, borderline west={3pt}{0pt}{shadecolor}, frame hidden, breakable, sharp corners]}{\end{tcolorbox}}\fi

Title

by

Vadym Yudenko

A thesis submitted in partial fulfillment of the requirements for the
degree of MA in Business and Financial Economics

Kyiv School of Economics

2023

Thesis Supervisor: Professor Tymofii Brik

Approved by\\
Head of the KSE Defense Committee, Professor {[}Type surname, name{]}

Date
\_\_\_\_\_\_\_\_\_\_\_\_\_\_\_\_\_\_\_\_\_\_\_\_\_\_\_\_\_\_\_\_\_\_\_

\hypertarget{acknowledgments}{%
\section{Acknowledgments}\label{acknowledgments}}

The author wishes to \_\_\_\_\_

\hypertarget{table-of-contents}{%
\section{Table of Contents}\label{table-of-contents}}

\hypertarget{list-of-figures}{%
\section{LIST OF FIGURES}\label{list-of-figures}}

Number Page

Figure 1.\\
Figure 2.\\
Figure 3.

\hypertarget{list-of-tables}{%
\section{LIST OF TABLES}\label{list-of-tables}}

Number Page

Table 1.\\
Table 2.\\
Table 3.

\hypertarget{list-of-abbreviations}{%
\section{LIST OF ABBREVIATIONS}\label{list-of-abbreviations}}

MCQ Monetary Choice Questionnaire

AFU Armed Forces of Ukraine

\hypertarget{chapter-1.-introduction}{%
\section{Chapter 1. Introduction}\label{chapter-1.-introduction}}

\emph{Background and motivation of this work.}

\begin{enumerate}
\def\labelenumi{\arabic{enumi}.}
\tightlist
\item
  How the full scale war has started and has an impact on Ukrainians
\end{enumerate}

Ukraine is experiencing a complex socioeconomic and political situation,
complicated by the ongoing conflict in Donbas, military events that
began on February 24, 2022, a total blackout, migration does not stop,
families have lost loved ones, housing - tens of thousands of people
were forced to move to other countries or safer regions. More than half
Ukrainians report that their financial situation has deteriorated
({``Прес-Релізи Та Звіти - Динаміка Самооцінки Матеріального Становища
Родини Після Російського Вторгнення: Лютий 2022 Року {\textendash}
Травень 2023 Року,''} n.d.; {``Правовий Захист Постраждалих Від Воєнних
Злочинів Росії (23-26 Грудня 2022),''} n.d.), some have lost have
everything: their home, business, livelihood. The income situation is
worse among residents of the south and east, middle-aged people, the
poor and low-income, as well as those who have lost their jobs or work
part-time. Now, the situation could become even worse if massive bombing
of the critical infrastructure continues this autumn.

``Every day, the number of people who will experience consequences for
their mental health will increase. Even those who were able to withstand
the first months of the war will suffer mental exhaustion because
getting used to being in a constant war can also harm mental health.
According to our preliminary forecаsts at the moment of the wаr, about
15 million Ukrаinians will need psychologicаl support in the future, of
which about 3-4 million will need to be prescribed medical treatment,''
said Minister of Health Viktor Liashko, presenting the results of the
express audit of services for the mental health care provided by state
authorities (Liashko, 2022).

Loss of life, loss of home, forced relocaion, economic losses due to
workers leaving\ldots{}

\begin{enumerate}
\def\labelenumi{\arabic{enumi}.}
\setcounter{enumi}{2}
\tightlist
\item
  Where does the discounting come in
\end{enumerate}

Ukrainians are living in unprecedented times. War has ravaged the
country and it consequences will be far reaching and long lasting. Some
of these consequence will have to do with people's attitudes towards
future. Living under constant threat of could make one live in the
moment and don't plan too much ahead, as the future might not come.
People value the present more over future. This phenomenon in behavioral
economics is called time discounting\footnote{In behavioral economic
  literature there are multiple names for the same phenomena: temporal
  discounting, delay discounting, intertemporal discounting. Present
  paper uses these interchangeably.} and it is linked with many life
outcomes. And the realities of Ukraine is that \textless migrants,
people becoming poorer, disabled, economy is destroyed, uncertainty is
as high as ever\textgreater{} all of which affect people's daily
choices. However we know that long-term growth require forward-looking
and more patient behavior.

The literature is clear on military conflict's effect on mental health
(Trujillo et al. 2021), financial situation and discounting (Imas, Kuhn,
and Mironova 2015)\ldots. But (Imas, Kuhn, and Mironova 2015) studies
only one aspect of temporal discounting - present bias. In our study the
phenomenon of temporal discounting will be studied through the lens of 4
temporal discounting types: magnitude effect, delay-speedup asymmetry,
gain-loss asymmetry, and present bias.

We want to study how war, through its many manifestations, affects
Ukrainian's temporal discounting.

\emph{Summarize the proposed approach (methodology you use) to the
problem and the results}

Given a vast complexity of war experiences, in order to establish the
effect of war we would have to consider multiple perspectives: influence
on mental health, financial situation. War doesn't directly affect
temporal discounting, it does so through other processes like perception
of one's safety, physical and mental health, financial
situation.\footnote{It is not ``Because of my war experience I started
  discounting future more'', but ``Because war affected my
  physical/mental health, and financial situation, I now value the
  present more (discount future more)''}

In this paper we combine tools from economics and psychology, namely MCQ
(Ruggeri et al. 2022), and SF-12 (Ware, Kosinski, and Keller 1996) to
analyze the relationship between potentially traumatic experiences and
temporal discounting. War related questions are taken from here
({``Fleeing Ukraine: Displaced People{'}s Experiences in the EU,''}
n.d.)

\emph{Paragraph about results}

A survey \_\_\_\_

\hypertarget{chapter-2.-literature-review}{%
\section{Chapter 2. Literature
review}\label{chapter-2.-literature-review}}

Discuss research papers related to your question in a structured way and
explain how you will contribute to the literature

\begin{enumerate}
\def\labelenumi{\arabic{enumi}.}
\tightlist
\item
  What is temporal discounting and which studies are relevant and why
\item
  Temporal discounting and mental health
\item
  Temporal discounting and financial situation
\item
  Temporal discounting and military conflict
\end{enumerate}

Time (temporal) discounting is the current relative valuation placed on
receiving a good or some cash at an earlier date compared with receiving
it at a later date (Loewenstein and Prelec, n.d.). This tendency is
often treated as a behavioral anomaly measured by presenting a series of
choices that vary values, timelines, framing (for example, gains or
losses) and other trade-offs. Responses to these series of lottery
questions can be aggregated or indexed in ways that test different
manifestations of the anomaly or the threshold at which individuals are
willing to change their preferences. Individuals vary in their responses
to intertemporal choice, and their revealed discount rates can be used
in predicting behavior across other domains. (which domains with
examples)

For example, people who discount future hypothetical monetary gains more
steeply, preferring smaller sums sooner to larger sums later, tend to
also discount non-monetary rewards: food (Duckworth, Tsukayama, and
Geier 2010), alcohol (MacKillop et al.~2010) and others. There are also
studies that find an association between favoring immediate rewards, as
in the classic ``marshmallow experiments'', with having poorer academic
performance, career success, and a higher likelihood of incarceration or
having a drug addiction. (Duckworth, Tsukayama, and Geier 2010; Eigsti
et al.~2006; Mischel, Shoda, and Rodriguez 1989).

There have been a vast literature on the effects of natural disasters on
choice preferences (Eckel, El-Gamal, and Wilson 2009; Cassar, Healy, and
Von Kessler 2017), and some relevant studies on effect of armed conflict
experience on, for example, risk aversion in South Korea (Kim and Lee
2014), social, risk and time preferences in Burundi (Voors et al.~2012),
and on happiness in Ukraine (Coupe and Obrizan 2016). Some studies
report that mental health affects discounting \_\_\_\_\_, others that it
doesn't Subjective mental health has been shown to be positively
associated with discounting (Löckenhoff, O'Donoghue, and Dunning 2011)
In the context of ongoing Russo-Ukrainian War, studying intertemporal
choices of people affected by war can help in creation of policy
decisions that will mitigate negative repercussions of exposure to
violence and destruction. This is important, because any effects
conflict might have on risk aversion, time discounting, or trusting
behavior, could substantially affect how people are able to recover from
conflicts.. In other words, exposure to violence impacts communities not
only through the direct losses they inflict, like physical damage or
seizure/destruction of business, but also through people's willingness
to attempt new business ventures, to save, and to work cooperatively. In
this research we want to study intertemporal choices of Ukrainians 1.5
years after the onset of full-scale invasion as it relates to experience
of war, media usage, and trust in media.

The extent to which individuals discount the future and whether they
discount in a time consistent fashion is an important determinant of
their life outcomes. The topic of war effects on mental, physical and
financial health has been thoroughly examined. (list papers) In
particular \ldots.

However there hasn't been enough papers focusing on how war affects
temporal discounting.

Why temporal discounting? Because it shows how people value the present
over the future. It has been found to be positively correlated with
\_\_\_\_\_, which is relevant for Ukraine.

\hypertarget{chapter-3.-methodology}{%
\section{Chapter 3. Methodology}\label{chapter-3.-methodology}}

Describe your approach in detail and explain why this methodology is
appropriate for the chosen research question

There are several channels through which war experience could affect
individual preferences. One plausible channel would be through a large
negative shock to wealth or income, which in turn would alter
preferences. A second channel, connected to the previous one is
increased uncertainty, which also affects preferences. A third channel,
from the psychology literature, would be through worsened mental and
physical health, which are often associated with higher discounting
rates. Any or all of these possibilities could affect individual's
preferences. A range of research gives the expectation that natural
disasters could affect preferences in these ways. Here, we review that
literature as we describe our predictions for the war's effects on
preferences.

\hypertarget{h1.-exposure-to-war-experiences-is-positively-associated-with-temporal-discounting}{%
\subsubsection{\texorpdfstring{H1. \ul{Exposure to war experiences} is
positively associated with temporal
discounting}{H1. Exposure to war experiences is positively associated with temporal discounting}}\label{h1.-exposure-to-war-experiences-is-positively-associated-with-temporal-discounting}}

Despite the prolonged , mobilizes people and has far-reaching
consequences

Experimental investigations have revealed two robust patterns in which
receipts and payments are not treated in the same way. One result, the
gain--loss asymmetry (Loewenstein \& Prelec, 1992)orthesign effect
(Thaler, 1981), is a main effect of outcome sign: People offer lower
interest rates for payments than they demand for receipts

Even people not directly affected by war, may suffer from its
consequences. Military violence doesn't just affect the mental health of
civilians living in conflict zones (Cesur, Sabia, and Tekin 2013), but
also of those who experience it through daily stressors such as changes
in physical health and financial situation, the destruction of social
networks and the mass displacement of the civilian population (Miller
and Rasmussen 2010).

Artefactual field experiments from zones of natural disaster or conflict
also suggest that exposed individuals' risk preferences are altered
(Eckel, El-Gamal, and Wilson 2009; Voors et al. 2012; Cassar, Healy, and
Von Kessler 2017)

In studying intertemporal choices during wartime there are multiple
things to keep in mind. First is that individual preferences are
affected in many ways by exposure to violence. For example, physical
damage to oneself or family/friends might have a different impact on the
person compared to damage to one's property. Even just witnessing
violence, like hearing sirens, explosions, seeing fighting can make
people believe that subsequent negative events will happen. A psychology
literature suggests that conflicts and natural disasters elicit strong
emotions, such as fear, apathy, and anger, which affect our decision
making. \textbf{(source)}

Finally, people may receive help from neighbors, strangers, the
government and non-governmental organizations, in addition to giving
help to others. All of these manifestations of war affect an
individual's choices and preferences. To understand which manifestations
can best explain the effect we turn to the relevant literature,
discussion of which follows below. Firstly, we look at the papers that
take a helicopter look on the topic of time discounting.

\hfill\break
(Loewenstein and Prelec, n.d.) lay the foundation of researching
anomalies of choice associated with Discounted Utility Theorem. They
systematize previous studies and provide methodology for assessing each
anomaly: magnitude, gain--loss asymmetry, delay--speedup asymmetry,
common difference and subadditivity. Drawing upon (Loewenstein and
Prelec, n.d.) paper (Ruggeri et al. 2022) tested generalizability of
temporal discounting in 61 countries using 5 choice anomalies. Key
findings that are relevant to us are: facing a negative environment
makes people prioritize immediate ``immediate clarity over future
uncertainty'' regardless of their income; temporal discounting scores
decrease with bigger wealth, except for the wealthiest individuals;
individual debt is associated with lower overall temporal discounting.
To understand how experience of an inhospitable environment affects
human decision-making across various domains we turn to the literature
on natural disaster and violent conflicts.

(Voors et al. 2012) conducted a series of field experiments in rural
Burundi to examine the impact of exposure to conflict on social, risk,
and time preferences. They find that individuals exposed to violence are
more altruistic, more risk-seeking, and have higher discounting.
However, in the end they report that the \textbf{net effect of exposure
to violence is unclear}, because on the one hand it encourages
risk-taking and increases the prosocial behavior, on the other hand it
also triggers impatience, which discourages savings and investments.

(Cassar, Healy, and Von Kessler 2017) study trust, risk, and time
preferences after a tsunami in Thailand. Similarly to (Voors et al.
2012), they find that the 2004 tsunami has led to substantial long
lasting increases in risk aversion, prosocial behavior, and impatience.
In assessing tsunamis impact they considered whether the village was
affected, whether the individual or household experienced financial
damage, and whether a family member was injured or killed. We will use a
similar approach in our paper.

\hypertarget{h2.-negative-financial-shock-is-positively-associated-with-discount-future}{%
\subsubsection{H2. Negative financial shock is positively associated
with discount
future}\label{h2.-negative-financial-shock-is-positively-associated-with-discount-future}}

Theoretical support for this hypothesis comes from

To assess financial situation of the population we would need to
consider both objective and subjective fin situaion. Objective being
reported monthly income, debts, vehicle subjective is measure with an
index. (M.O. and Ya.Ye. 2021)

\hypertarget{h3.-poor-mental-health-is-associated-with-higher-discouting}{%
\subsubsection{H3. Poor mental health is associated with higher
discouting}\label{h3.-poor-mental-health-is-associated-with-higher-discouting}}

Theoretical support for this hypothesis comes from

To study these hypotheses we conduct a survey largely adapted from
multiple papers. Questions on temporal discounting are adapted from
(Ruggeri et al. 2022), mental health - (Ware, Kosinski, and Keller
1996), and war experiences - ({``Fleeing Ukraine: Displaced People{'}s
Experiences in the EU,''} n.d.). First respondents answer a set of
adaptive MCQ, then questions on financial situation, followed questions
about mental health, war experiences, and finish with a set of
socio-demographic questions.

Below, we describe each of our experimental treatments in detail.
Complete experimental instructions can be found in the Appendix.

\hypertarget{measuring-temporal-discounting}{%
\subsection{3.1 Measuring Temporal
Discounting}\label{measuring-temporal-discounting}}

To measure temporal discounting and its anomalies we mostly follow
Ruggeri et al. (2022).

The primary measures involve discrete choices between an immediate gain
or payment 2000 UAH and a delayed gain or payment 2600 UAH of either now
or in 3 months. All participants see the same baseline items that are
worth approximately 10\% of the average monthly household income.

Respondents were asked to choose between a smaller-sooner (SS) outcome
and larger-later (LL) one:

Would you rather\ldots{}

\begin{itemize}
\item
  Get 2000 UAH right now
\item
  Get 2600 UAH in 3 months
\end{itemize}

If you had to pay for something, would you rather\ldots{}

\begin{itemize}
\item
  Pay 2000 UAH right now
\item
  Pay 2600 UAH in 3 months
\end{itemize}

Someone who is indifferent between SS and LL for gains (demanding a 30\%
interest rate) will typically prefer SS for losses(offering less than
30\%). In constructing these questions several aspects were addressed.

Subsequent items change the delayed option based on respondent's initial
choice, increasing for those who chose the immediate gain, decreasing
those that prefer the delayed prospect, and vice-versa for
losses/payments. Individuals will be given a score, depending on their
answers. In the end we will have a score that will reflect an
individual's preferences. This strategy results in a score of zero for
individuals always choosing the delayed amount and 10 for individuals
choosing the sooner reward.

Participants must differentiate between the options, in other words they
should feel that these are 2 different options. Initial value the person
can get/pay right now must be \textbf{salient}: for a millionaire 2000
is no different from 2600, but offer this amount to a student and they
will see that these options are different.

For each outcome valence (gain or loss), the immediate amount of 2000
UAH was held constant across trials. There were seven levels of delayed
amounts (2000, 2400, 2500, 2600, 2700, 2800) presented at four delays
(7, 30, 90, and 180 Days)

Benchmark was set to 2000 UAH; discount rate of 30\% which reflects
annual inflation rate; and a time frame of ``now-3 months''. The survey
goes as follows.

After MCQ, participants then answer seven items regarding general
financial risk and personal economic conditions that have been useful in
other behavioral research on choices and economic inequality, and six
demographic items: age, education, location. There is also a section on
experience of war, before answering which participants were warned that
questions might be sensitive and they may not give an answer ``Don't
want to answer''. Lastly, there is a section on social media usage and
trust in sources of information. The survey was in Ukrainian and ran for
2 weeks. During this time it had collected 73 responses from Ukrainians
in Ukraine and abroad. Participants were found via convenience sampling
by posting about the survey in social media, which is not ideal for this
type of study, because it can lead to biased and not representative
samples. This being our pilot study we are content with this kind of
sample. In the final paper this will be resolved.

Temporal discounting was assessed with a task adapted from Mitchell
(1999) and implemented on a laptop computer using E-prime (version 2.0)
experimental software. Participants were asked to make a series of
choices between a smaller outcome affecting them immediately and a
larger outcome affecting them at some time in the future. For example,
they saw the options ``gain \$5 now'' and ``gain \$7.50 in 90 days'' and
used the computer keyboard to select one of the options. Each choice was
either a choice between an immediate gain versus a delayed gain
(henceforth referred to as the ``gain condition'') or a choice between
an immediate loss versus a delayed loss (``loss condition''). For each
outcome valence (gain or loss), the immediate amount of \$5 was held
constant across trials. There were seven levels of delayed amounts
(\$4.75, \$5.25, \$5.50, \$6.00, \$6.50, \$7.00, \$7.50) presented at
four delays (7, 30, 90, and 180 Days). Each combination of amounts and
delays was presented once in the gain condition and once in the loss
condition.

Psychologists have discovered numerous anomalies to a rational choice
analysis, in which the interest or discount rate implied by choice is
constant and equal to the decision maker's opportunity cost of money.
For many people, this is the bank rate of interest or the mortgage rate.

\hypertarget{measuring-financial-wellbeing}{%
\subsection{3.2 Measuring Financial
Wellbeing}\label{measuring-financial-wellbeing}}

with financial wellbeing we want to see

To measure financial well-being, we use both objective and subjective
aspects. Objective FWB is measured as the real total household income --
the sum of all household income sources net of taxes and deflated by the
consumer price index. To evaluate subjective FWB, we use two
self-reported variables. The first variable is the ``ability to make
ends meet'' (``AEM''), which is very commonly used to measure financial
well-being (Angel et al., 2003; Arber et al., 2014). It takes a value of
1 if the person had enough income to buy food and 0 otherwise. RLMS does
not contain questions that can be used to measure this ability to make
ends meet. Thus, for the Russian sample we use a variable that measures
``financial satisfaction'', which is also used as an indicator of
subjective financial well-being in other studies (i.e.~Zimmerman, 1995).
It takes a value of 1 if the person is satisfied and 0 otherwise. To
evaluate how secure a person feels about their financial future, we use
the self-reported ``expectation about financial well-being'' (``FWB'').
It takes a value of 1 if the person expects improvements and 0
otherwise. More detailed information about these measures is presented
in Table A1in Appendix A. We distinguish between subjective FWB and
income, because income alone does not reflect the ability to meet one's
needs (Zimmerman and Katon, 2005) and has a different dynamic to
expectations about the financial situation (Angel et al., 2003; Joo and
Grable, 2004). Current real income varies with short-term fluctuations
in prices and economic activity.

Expectations are more sensitive to the arrival of new information about
the conflict (including the influence of fake news, information warfare,
etc.) and may shape the investment decisions of individuals. This
naturally has implications for policymakers. Debts are related to
intertemporal discounting (Ikeda and Kang 2015)

\hypertarget{measuring-mental-health}{%
\subsection{3.3 Measuring Mental Health}\label{measuring-mental-health}}

As per the findings of various academic publications, it has been
established that the impact of war on mental health and well-being is
notably negative, resulting in the manifestation of a range of
psychological disorders such as PTSD, depression (Charlson et al., 2019;
Gonçalves Júnior et al., 2022; Hendrickx et al., 2019; Osokina et al.,
2022; Perkins et al., 2018), and suicidal ideation (Gonçalves Júnior et
al., 2022)

Studies of the health of Ukrainian children and adolescents have
confirmed that the risk of developing symptoms of depression exists both
during the war and in the post-war period (Gonçalves Júnior et al.,
2022; Osokina et al., 2022) According to Gonçalves Júnior et al.~(2022),
military conflicts also worsen the psycho-emotional state of
war-affected populations, resulting in negative feelings such as
sadness, anger, guilt, or loneliness.

As there are no feasible ways to measure objective health, we measure
subjective health with SF-12 (Ware, Kosinski, and Keller 1996), a widely
used and well-validated measure which yields separate scores for mental
and physical health. Multiple-choice and Likert-type items require
participants to rate their general health as well as aspects of mental
well-being and functional health. This index was used in relevant
research and has shown good results (Löckenhoff, O'Donoghue, and Dunning
2011)

Regarding health, we make a distinction between physical and mental
health, because we expect the conflict's effect on them to be subject to
different mechanisms, timing and duration. The effect on physical health
may develop later, but with longer and more severe effects, leading to
disabilities and chronic diseases (Alderman et al., 2006; Palmer et al.,
2016). Conversely, the effect on mental health is more acute in the
short run, but the longer the conflict continues, the more likely it is
that people will adjust to their new situation (Shemyakina and Plagnol,
2013; Miller and Rasmussen, 2010). Previous studies point out that the
spread of communicable diseases arising from forced migration flows
during conflict increases a number of chronic diseases and disabilities
(Toole and Waldman, 1997; Roberts et al., 2017). As an indicator of
changes in physical health, we use self-reports as to whether a person
had a chronic disease within the previous 12 months. To measure mental
health, we employ self-reports about stress symptoms within the previous
12 months. Physical responses to stress can be reflected in a set of
symptoms (anxiety, depression, insomnia, migraine, etc.; Farhood et al.,
1993). The dependent variable for the basic model is constructed based
on the responses to a set of questions concerning whether the person had
any symptoms of stress. To avoid difficulties with correspondence and to
have a reasonable share of cases, all questions that measure any
symptoms of stress (namely, frequent headaches and migraine, depression
and anxiety, and hypertension) are aggregated into one category --
stress symptoms -- as the simple average of the three components. More
detailed definitions of the health variables are given inTable A1in
Appendix A.

To sum up, it shall be stated that in the case of Ukrainian war
refugees, all of the above-mentioned risk factors may occur. The
Ukrainians fleeing the war are exposed to war trauma, evacuation trauma,
acculturation trauma and media trauma (McDonnell et al., 2022; Rizzi et
al., 2022; Javanbakht, 2022). Stress among refugees is increased by the
uncertainty about their future (Newnham et al., 2019). Part of the
Ukrainian refugees experienced previous trauma which had been given rise
by the necessity to leave their former place of residence as a result of
Russian military activities taking place in the east of Ukraine since
2014 (Johnson et al., 2021 ; Długosz et al., 2022 ). Trauma is observed
among both internally and externally displaced refugees (Leon et al.,
2022).

\hypertarget{measuring-war-experience}{%
\subsection{3.4 Measuring War
Experience}\label{measuring-war-experience}}

More intense and more recent exposure to violence are more likely to
generate a triggerability in risk preference.

The problem of the well-being and mental health of societies during a
period of crisis -- in particular, war -- is currently viewed and
reported on in a vast body of literature from a variety of perspectives.
Specific attention is addressed towards physical well-being (Osiichuk
and Shepotylo 2020) (Cheung et al., 2020; Jahanshahi et al., 2020;
Osiichuk \& Shepotylo, 2020), mental wellbeing (Jahanshahi et al., 2020;
Kurapov et al., 2022c; Mohd Saleem et al., 2021; Pavlenko et al., 2022;
Rizkalla \& Segal, 2018), financial well-being (Osiichuk \& Shepotylo,
2020), and social well-being (Barchielli et al., 2022; Bragin et al.,
2021; Jahanshahi et al., 2020; Kurapov et al., 2022; Kurapov et al.,
2022a). As reported, physical well-being has two dimensions: physical
welfare (Cheung et al., 2020) and physical health (Jahanshahi et al.,
2020). In particular, it has been found that public access to physical
goods, such as housing, food, etc., significantly declines during
war-related crises, as has been confirmed by exploring the welfare and
quality of life of those affected by the Syrian conflict (Cheung et al.,
2020). The impact of military conflicts on physical health is evident
but not comprehensively covered in the literature, even though
researchers highlight the negative relationship between these variables
(Jahanshahi et al., 2020; Osiichuk \& Shepotylo, 2020). Much attention
is dedicated to the adverse impact of war and other related crises on
overall well-being and metal health -- in particular, financial health,
which suffers significantly. Researchers outline the general
deterioration of the financial well-being of citizens in short- and
long-term perspectives (Osiichuk \& Shepotylo, 2020), and the lack of
proper access to regular social benefits (Hendrickx et al., 2019). Other
negative consequences of war include the deterioration of general social
well-being (Bragin et al., 2021; Jahanshahi et al., 2020), the reduction
of social support (Cheung et al., 2020), the inability to maintain
previously-established social relationships, the use of habitualized
behavioral coping strategies (Kurapov et al., 2022a), and the emergence
and further maintenance of discrimination against social minorities and
other vulnerable social groups (Mohd Saleem et al., 2021).

In order to meaningfully measure experience of war it is important to
break it down into 2 categories: objective and subjective. (Papadopoulos
2021) also about involuntary dislocation

The prevailing mental health issue during the post-war period is
post-traumatic stress disorder (PTSD) (references). While trauma can be
a devastating experience, it is important to recognize that it can also
be an opportunity for growth and transformation. PTG is often associated
with the concept of resilience, which refers to the ability to bounce
back from adversity. While resilience is an important trait to have, PTG
takes it one step further by suggesting that individuals can actually
benefit from traumatic experiences in unexpected ways (Dell'Osso et al.,
2022; Jin et al., 2014). However, not everyone is affected by war the
same way: some bounce back and exhibit positive psychological changes
that can occur after experiencing a traumatic event, which are referred
to as Post Traumatic Stress (PTG). It was found that people living in
war-torn areas exhibit moderate levels of PTG. Women are more prone to
PTG than men; younger and older participants show higher levels of PTG,
while middle-aged participants exhibit lower levels; and financial
security increases PTG. Presence in Ukraine increases personal strength,
while living outside of Ukraine increases the possibilities for new PTG
strategies. Trauma exposure during the war does not increase levels of
PTG Post-Traumatic Growth (PTG) (``FROM TRAUMA TO TRANSFORMATION:
PREDICTORS OF POST-TRAUMATIC GROWTH IN UKRAINIANS AFFECTED BY WAR IN AN
ONGOING CONFLICT SETTING,'' n.d.) Objective experience is what the
person has actually experienced: forced relocation, property damage,
loss of a friend/family member, physical damage\ldots{} - something that
has actually happened. And might and mightn't affect the person. There
exists multiple measures of . One frequently used is The Life Events
Checklist for DSM-5 (LEC-5) - a self-report measure designed to screen
for potentially traumatic events in a respondent's lifetime. There are
also WEQ (Karam et al. 1999) These questionnaires involve the usual
combat experiences (being ``under the enemy fire'', ``surrounded by the
enemy'', ``witnessed killing of others'', etc.) like in the Combat
Exposure Scale with 7 questions19; witnessing of bizarre death or
mutilation (Military Stress Scale with 6 questions 20; Combat Exposure
Index with 7 questions21); killing of civilians (another instrument
under the same name -- Combat Exposure Scale with 7 questions22). Some
questionnaires (Vietnam Era Stress Inventory23, Women's Wartime Stressor
Scale24) comprise questions about unpleasant experiences in the military
setting (sleep and/or food deprivation, lack of support, reproach and
offences upon return home). Most of the above questionnaires represent
the brief scales assessing the stressors specific to war zone and some
of them have good measures of reliability (as measured by Cronbach's
alpha) A special problem in assessing stressor exposure is accuracy and
authenticity of the retrospective reports provided by the subjects. The
researcher's goal is to `'minimize the probability that client's reports
are incorrect and maximize the likelihood of the completeness of his
reports' One thing we can't neglect is frequency and intensity of
exposure. People who have experienced sirens in Zakarpaska region have
had much less of an exposure than those in Zaporizhzha. In addition to
those in Zakarpatska , the siren is merely a warning to them, while
those living closer to the frontline might view sirens as a threat.
However this won't fully capture the effect these events had on the
person. For this we must measure how these events affected the person in
question. Since objectively terrible events can have different effects
on people, some, despite the suffering they've endured reassess their
life and report growth (Post-Traumatic Growth (PTG)), however with this
measure we can't say what their condition were before the war and
wheather this growth

\hypertarget{control-variables}{%
\subsection{3.4 Control Variables}\label{control-variables}}

\textless some words on socio-demographic questions\textgreater{}

\hypertarget{chapter-4.-data}{%
\section{Chapter 4. Data}\label{chapter-4.-data}}

The questionnaire form included questions that concerned
sociodemographic information, trauma exposure, and \_\_\_\_.

EDA and descriptive statistics of the experiment

\hypertarget{chapter-5.-results}{%
\section{Chapter 5. Results}\label{chapter-5.-results}}

The data collected during the study were subjected to statistical
processing using \_\_\_\_\_\_ methods of analysis. Statistical analysis
was performed using R (version \textbf{\emph{). The study included}}
participants (age M = \emph{\textbf{) --} males and} \_\_ females. For
estimating the size of the effect of age, we outlined the following age
groups: youth (N = , M = , age range --), young adults (N = , M = , age
range --), adults (N = , M = , age range --), and middle-aged (N = , M =
, age range --).

5.1.

5.2.

5.3.

\hypertarget{chapter-6.-conclusions-and-recommendations}{%
\section{Chapter 6. Conclusions and
Recommendations}\label{chapter-6.-conclusions-and-recommendations}}

\hypertarget{ethics-approval-and-informed-consent}{%
\section{Ethics approval and informed
consent}\label{ethics-approval-and-informed-consent}}

\hypertarget{references}{%
\section*{References}\label{references}}
\addcontentsline{toc}{section}{References}

\hypertarget{refs}{}
\begin{CSLReferences}{1}{0}
\leavevmode\vadjust pre{\hypertarget{ref-cassar2017}{}}%
Cassar, Alessandra, Andrew Healy, and Carl Von Kessler. 2017. {``Trust,
Risk, and Time Preferences After a Natural Disaster: Experimental
Evidence from Thailand.''} \emph{World Development} 94 (June): 90--105.
\url{https://doi.org/10.1016/j.worlddev.2016.12.042}.

\leavevmode\vadjust pre{\hypertarget{ref-cesur2013}{}}%
Cesur, Resul, Joseph J. Sabia, and Erdal Tekin. 2013. {``The
Psychological Costs of War: Military Combat and Mental Health.''}
\emph{Journal of Health Economics} 32 (1): 51--65.
\url{https://doi.org/10.1016/j.jhealeco.2012.09.001}.

\leavevmode\vadjust pre{\hypertarget{ref-eckel2009}{}}%
Eckel, Catherine C., Mahmoud A. El-Gamal, and Rick K. Wilson. 2009.
{``Risk Loving After the Storm: A Bayesian-Network Study of Hurricane
Katrina Evacuees.''} \emph{Journal of Economic Behavior \& Organization}
69 (2): 110--24. \url{https://doi.org/10.1016/j.jebo.2007.08.012}.

\leavevmode\vadjust pre{\hypertarget{ref-fleeing}{}}%
{``Fleeing Ukraine: Displaced People{'}s Experiences in the EU.''} n.d.

\leavevmode\vadjust pre{\hypertarget{ref-ikeda2015}{}}%
Ikeda, Shinsuke, and Myong-Il Kang. 2015. {``Hyperbolic Discounting,
Borrowing Aversion and Debt Holding: Hyperbolic Discounting and Debt
Holding.''} \emph{Japanese Economic Review} 66 (4): 421--46.
\url{https://doi.org/10.1111/jere.12072}.

\leavevmode\vadjust pre{\hypertarget{ref-imas2015}{}}%
Imas, Alex, Michael Kuhn, and Vera Mironova. 2015. {``A History of
Violence: Field Evidence on Trauma, Discounting and Present Bias.''}
\emph{SSRN Electronic Journal}.
\url{https://doi.org/10.2139/ssrn.2603650}.

\leavevmode\vadjust pre{\hypertarget{ref-karam1999}{}}%
Karam, E. G., R. Al-Atrash, S. Saliba, N. Melhem, and D. Howard. 1999.
{``The War Events Questionnaire.''} \emph{Social Psychiatry and
Psychiatric Epidemiology} 34 (5): 265--74.
\url{https://doi.org/10.1007/s001270050143}.

\leavevmode\vadjust pre{\hypertarget{ref-loewenstein}{}}%
Loewenstein, George, and Drazen Prelec. n.d. {``ANOMALIES IN
INTERTEMPORAL CHOICE: EVIDENCE AND AN INTERPRETATION.''} \emph{QUARTERLY
JOURNAL OF ECONOMICS}.
https://doi.org/\url{https://doi.org/10.2307/2118482}.

\leavevmode\vadjust pre{\hypertarget{ref-miller2010}{}}%
Miller, Kenneth E., and Andrew Rasmussen. 2010. {``War Exposure, Daily
Stressors, and Mental Health in Conflict and Post-Conflict Settings:
Bridging the Divide Between Trauma-Focused and Psychosocial
Frameworks.''} \emph{Social Science \& Medicine} 70 (1): 7--16.
\url{https://doi.org/10.1016/j.socscimed.2009.09.029}.

\leavevmode\vadjust pre{\hypertarget{ref-osiichuk2020}{}}%
Osiichuk, Maryna, and Oleksandr Shepotylo. 2020. {``Conflict and
Well-Being of Civilians: The Case of the Russian-Ukrainian Hybrid
War.''} \emph{Economic Systems} 44 (1): 100736.
\url{https://doi.org/10.1016/j.ecosys.2019.100736}.

\leavevmode\vadjust pre{\hypertarget{ref-papadopoulos2021}{}}%
Papadopoulos, Renos K. 2021. \emph{Involuntary Dislocation: Home,
Trauma, Resilience, and Adversity-Activated Development}. 1st ed.
Routledge. \url{https://doi.org/10.4324/9781003154822}.

\leavevmode\vadjust pre{\hypertarget{ref-ruggeri2022}{}}%
Ruggeri, Kai, Amma Panin, Milica Vdovic, Bojana Većkalov, Nazeer
Abdul-Salaam, Jascha Achterberg, Carla Akil, et al. 2022. {``The
Globalizability of Temporal Discounting.''} \emph{Nature Human
Behaviour} 6 (10): 1386--97.
\url{https://doi.org/10.1038/s41562-022-01392-w}.

\leavevmode\vadjust pre{\hypertarget{ref-trujillo2021}{}}%
Trujillo, Sandra, Luz Stella Giraldo, José David López, Alberto Acosta,
and Natalia Trujillo. 2021. {``Mental Health Outcomes in Communities
Exposed to Armed Conflict Experiences.''} \emph{BMC Psychology} 9
(August): 127. \url{https://doi.org/10.1186/s40359-021-00626-2}.

\leavevmode\vadjust pre{\hypertarget{ref-voors2012}{}}%
Voors, Maarten J, Eleonora E. M Nillesen, Philip Verwimp, Erwin H Bulte,
Robert Lensink, and Daan P. Van Soest. 2012. {``Violent Conflict and
Behavior: A Field Experiment in Burundi.''} \emph{American Economic
Review} 102 (2): 941--64. \url{https://doi.org/10.1257/aer.102.2.941}.

\leavevmode\vadjust pre{\hypertarget{ref-ware1996}{}}%
Ware, John E., Mark Kosinski, and Susan D. Keller. 1996. {``A 12-Item
Short-Form Health Survey: Construction of Scales and Preliminary Tests
of Reliability and Validity.''} \emph{Medical Care} 34 (3): 220--33.
\url{http://www.jstor.org/stable/3766749}.

\leavevmode\vadjust pre{\hypertarget{ref-pravoviy}{}}%
{``Правовий Захист Постраждалих Від Воєнних Злочинів Росії (23-26 Грудня
2022).''} n.d.
\url{http://ratinggroup.ua/research/ukraine/pravoviy_zahist_postrazhdalih_v_d_vo_nnih_zlochin_v_ros_23-26_grudnya_2022.html?fbclid=IwAR02qqn1w4sou_EQ-KUc9vCpU9ejW3hnkLhaTN1aKdJC6wdSng1tj3JKAp0}.

\leavevmode\vadjust pre{\hypertarget{ref-pres-rel}{}}%
{``Прес-Релізи Та Звіти - Динаміка Самооцінки Матеріального Становища
Родини Після Російського Вторгнення: Лютий 2022 Року {\textendash}
Травень 2023 Року.''} n.d.
\url{https://kiis.com.ua/?lang=ukr\&cat=reports\&id=1256\&page=1}.

\end{CSLReferences}

\hypertarget{appendix-a}{%
\section{Appendix A}\label{appendix-a}}

Survey can be found via link

\hypertarget{leftovers}{%
\section{Leftovers}\label{leftovers}}

(Coupe and Obrizan 2016) look at how war affected happiness in Ukraine.
This paper doesn't study financial decisions and intertemporal
discounting, however it was useful in approaching questions about
experience of war. Another paper by the same authors (Coupé and Obrizan
2016) which studies violence and political outcomes in Ukraine allows us
to categorize damage caused by war. Knowing the potential effects
exposure to violence has on human decision making we needed now to apply
this insight in our survey, specifically in the Monetary Choice
Questionnaire. For this we need to figure out a benchmark - which
monetary amount will be salient for our responders, a discount rate, and
a time frame. There are many studies that find various discount rates
for different time periods and amounts (see (Frederick and Loewenstein
2002) for a review). In general, shorter time periods are associated
with higher rates of observed discounting. (Ruggeri et al.~2022) uses
10\% of median salary as a benchmark, 10\% discounting rate and ``now-12
months-24 months'' time frame. While this study involved Ukraine in
2021, using the same percentage and time will not yield valuable results
for us. To understand conditions facing Ukrainians right now, we turn to
a paper on Argentina and Britain (Macchia, Plagnol, and Reimers 2018),
in which authors use inflation rate in both countries (Britain - low,
Argentina - high inflation) to study people's sensitivity with regards
to delay discounting choices. Another study in Zimbabwe (Larochelle,
Alwang, and Taruvinga 2014) provides an extreme account of
hyperinflation and how it affects people's wellbeing. As the
Russo-Ukrainian war is the most televised war of recent years, we
thought about looking into the use of social media and how it relates
with discounting. (Shenhav, Rand, and Greene 2017) in their study of
temporal discounting found that people who with steeper discounting
prefer to consume information that is less complex and multi-faceted,
and use short-form social media (e.g.~Twitter). Our contribution will
lie in explaining how various experiences of war affect people's
discounting and financial decision making.



\end{document}
